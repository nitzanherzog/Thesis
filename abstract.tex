%% Set the space above the title appropriately, for the amount of text
%% that you have in your abstract
\vspace*{1.5cm}
%%\begin{center}

{\Huge\textbf{Abstract}}
%%\end{center}

\vspace*{1.5cm}
%%\normalsize
%\begin{abstract}

Volume transmission is a communication modality mediated by diffusion of signalling species in the extrasynaptic space. This mode of communication is thought to be significantly involved in cognitive processing but nevertheless remains poorly understood and its has been hampered by the lack of experimental tools for faithfully recreating the spatiotemporal patterns of these processes in a controlled manner. Microfluidics technology, which allows rapid and precise manipulation of fluid at the micron scale is ideally suited for generation of precise spatiotemporal patterns of chemical species. Past microfluidics work applied to neuroscience has mainly focused on generating slow spatial gradients but have not accounted for signals occurring in rapid time scales such as in the neuromodulatory systems where agonist pulses lasting just a few seconds are generated. Producing such rapid signals using microfluidic technology requires rapid flow rates but it is unknown how such flow would affect neurons both functionally and in terms of viability.

In this work, we monitored the viability and spiking activity of neuronal culture under rapid microfluidic flow in custom devices with integrated microelectrode arrays (MEAs). We found that rapid flow can cause degeneration and disfunction that are mediated primarily by a removal of conditioning factors from around the cells whereas we did not observe an effect of shear. We were further able to establish flow conditions permissive to proper network function which may be achieved either by reducing the the factor removal flux or by matching the chemistry of the flow media to the microenvironment around the cells.

After establishing a working flow protocols we proceeded to construct a microfluidic system for generating agonist transients to an entire neuronal microculture grown on MEAs. Since such microcultures are not a standard neuroscience preparation, we demonstrated that they develop normally and exhibit useful spontaneous and evoked activity. We further demonstrated that the agonist transients adhere to physiological time scales through a visualization of the pulse action coupled to a finite element model and through a direct measurement of the cultures' excitation under pulses of glutamate. Finally, we conducted experiments where dopamine pulses were coupled to electrical stimulations. These experiments were designed to validate a classical computational theory predicting how synaptic plasticity may be gated by dopamine to generate effective reinforcement learning. Although we did not observe plasticity, we found a direct modulation of the network activity by the presence of dopamine, which confirms that the cultures possess dopamine machinery and are therefore useful for interrogating related questions. This system will be useful in validating computational models of neuromodulation and activity which have so far lacked an experimental partner.

In the wider perspective, we expect that the presented system and the provided data will lead the way to microfluidic systems with an ability to control increasingly complex aspects of the neuronal microenvironment and hence to an improved understanding of the role of volume transmission in neural homeostasis and information processing.

%\end{abstract} 