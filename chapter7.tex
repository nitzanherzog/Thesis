\chapter{Summary and discussion}
\label{chap:Discussion}


\section{Establishment of a neuronal culture on MEAs based model}
In chapter \ref{chap:activity}, in accordance with the first goal of this work, we developed a model of activity in neuronal ensembles based on mouse cortical culture grown on MEAs. Such systems have been widely used since substrate integrated MEAs were first fabricated in the 1980's. However, these past utilizations have been almost exclusively restricted to cultures derived from rat embryos. Consequently, the data provided in this work is interesting with respect to the possibility of using mouse neurons with MEAs which would be desirable on account of the richer molecular biology toolbox available for that model animal. We have shown that mouse cultures indeed develop into an active network and we were able to record both spontaneous and evoked multi channel spiking activity from them. Additionally, the cultures demonstrated several characteristics originally described in the rat systems (reviewed in section \ref{sec:introduction:MEANetwork} and also mentioned in section \ref{sec:activity:activityStats}). These include a gradual formation of synaptic connections accompanied by a reorganization of the activity into synchronized network bursts, a clear homeostatic control over firing rates, a modular connectivity organization and a sharpening and narrowing of the bursts over time. This last observation is taken as evidence that the GABAergic system follows a distinct developmental time course compared to its glutamatergic counterpart. Based on these observations we conclude that mouse cultures grown on MEAs are a useful system for studying the dynamics of neuronal ensembles at the network level.

Attempts to induce plasticity in the mouse cultures using a tetanus-based protocol and without addition of any neuromodulators did not produce any significant changes to measures involving either spontaneous or evoked activity. This negative result joins a long list of contradicting reports regarding plasticity in neuronal cultures. Nevertheless, it aligns well with the accepted notion that neuromodulators strongly gate neuronal plasticity (reviewed in section \ref{sec:introduction:neuromodulators}). Thus we regard this negative result as an added motivation to supplement classical cortical cultures with neuromodulator signalling functionality. In this chapter, we also took the extra step of attempting the same plasticity protocol with dopamine present in the media during the tetanus induction epoch and subsequently washed away. The reason behind this particular pattern of application is that dopamine (and neuromodulators in general) is known to have both a direct effect on network dynamics as well as a plasticity effect which persists after the agonist had been removed. Since the experiment was concerned with plasticity, we elected to remove the dopamine prior to the post-induction measurements so that the plasticity effect would not be confused with the direct one. This addition of dopamine represents a novel paradigm that had not been explored in neuronal culture before and interestingly resulted in significant changes to both the evoked and spontaneous activity measures. This result serves as another demonstration that neuromodulators constitute an important part of the system and they are required to observe the full range of forebrain circuit dynamics. Unfortunately, not much may be taken from these results regarding the precise interactions of dopamine with network and synaptic activity because such manual pipetting of the agonist on and off is not conducive to fine tuned associations between these processes. Additionally, this application method does not assure that the dopamine has been completely removed (because enough media from the same culture dish is available for just a single washing step) and might also induce activity effects related to the solution exchange itself. Thus these results highlight the current experimental need for a high precision programmatic solution exchange system operating under continuous flow (to avoid effects associated with the solution exchange itself) and support the endeavour undertaken in this Ph.D work.

As described above, the mouse cultures were shown to develop a characteristic network activity profile and thus to be useful in studying activity dynamics and plasticity. Nevertheless, they seemed to follow a different developmental time course compared to reports from rat cultures. This was mainly to do with a delay in synaptic maturation. In the case of our mouse cultures, the appearance of network activity was clearly characterized by an initial time frame where the MEA channels showed tonic firing with very low levels of correlation and virtually no network bursts. Significant correlations arose only several days later. According to developmental studies of rat cultures the activity in those cases is characterized by significant correlations immediately upon its appearance (see cited references in section \ref{sec:activity:activityStats}). It also seems to be more influenced by synaptic drive because the mean firing rate and the burst rate are correlated and increase until the 4\textsuperscript{th} week \textit{in vitro}. In the case of the mouse cultures, even after the appearance of correlations, the activity levels remained constant implying a dominance of intrinsic excitability and homeostasis over synaptic drive. The delayed maturation in mouse cultures is also manifested by the complete lack of superbursts, which were abundant in rat cultures. These high intensity network events are considered to result from a temporary imbalance between excitation and inhibition present in early phases of development in rat, but apparently not mouse, cultures (This argument is expanded in section \ref{sec:activity:activityStats}). Since the development of neuronal culture is considered to represent, to a certain extent, the cortical development just following birth, the above noted differences might reflect a distinction between how the nervous system initially organizes in mouse and in rat. We find it is surprising that such similar species exhibit such marked differences. Whether these differences serve a functional purpose or merely comprise a genetic jitter which will be overridden by experience dependent organization which occurs later in development remains an open question.

\section{Establishment of a microfluidics based neuronal culture system}
In chapter \ref{chap:devicesAndFlow}, in accordance with the second goal of this work, we were able to establish both long term macro- and micro-cultures in microfluidic devices of the relevant geometries. We found that, in accordance with reports from literature, cultivating such cultures was challenging as compared to standard ones on open surfaces and required tweaking of the culture, surface and incubation parameters as well as treating of the device materials (PDMS extraction). The micro-cultures were particularly challenging and developed properly only at high densities and when the device design was modified to include support cultures enriching the media with conditioning factors.

The cultures were able to develop inside the microfluidic devices for over 3 weeks when kept in static conditions. However, maintaining their viability under flow proved to be extremely tricky as it caused them to degenerate within several hours. This degeneration occurred over the entire flow range available to our flow system. A quantitative analysis of the degeneration showed that by using conditioned media for the flow, the degeneration was slowed down significantly and that the rate of degeneration was highly correlated with the levels of conditioning. Varying the flow rate across two orders of magnitude did not, however, correlate with the viability in any way. To explain these results we propose that that the deleterious effects of flow are mediated primarily by removal of conditioning factors. based on this premise, conditioned media extends the viability by partially replacing the removed factors. The lack of sensitivity to flow rate is explained by the fact that even minuscule flow is enough to compete with diffusion and remove cell-secreted factors so it is plausible that any level of convection dominant flow would induce the same effect. One way of assessing the dominance of convection is by comparing the average flow velocity to the expected translation of a particle per time unit given a diffusive regime. Thus, for example, a neurotransmitter sized molecule would traverse \(x=2\sqrt{t\cdot D}=40\mu m\) in 1 second with \(D=400 \mu m^{2}\cdot s^{-1}\) \cite{johnstoneThesis} purely due to diffusion. In this sense, our lowest flow velocity (\(40 \mu m\cdot s^{-1}\)) is comparable to diffusion of small neurotransmitter molecules. However, in the literature, the convection-diffusion balance is normally expressed using a different measure, the Peclet number. This measure formally refers to the length of travel down a microfluidic channel that is required for two laminar streams flowing in it side by side to achieve full mixing via diffusion \cite{squires2005microfluidics}. The Peclet number provides this length as multiples of channel width (note that it is a dimensionless measure). This number is generally used to quantify the convection-diffusion balance even though it does not have a formal theoretical meaning in some of the contexts, such as the question of removal of secreted factors by convection which is addressed here. The Peclet number for our slowest flow regime is \(P_{e}=\frac{L\cdot u}{D}=150\) where \(L=1500\mu m\) is the channel width and \(u=40 \mu m\cdot s^{-1}\) is the average velocity. This value is still somewhat greater than in previous studies where long term neuronal culturing under flow was achieved (see sections \ref{sec:introduction:neuroFlow} and \ref{sec:devices:introduction}) where the Peclet number was \(P_{e}<50\). This could explain the fact that even in our slowest flow regime, the cultures' degeneration rate was greater than control. Another important detail regarding these previous studies is that the flow in all of them was gravity fed and fully on-chip which means that it was very smooth. It is important to take into account that our flow system may be characterized by a more jerky type of flow because of oscillations originating in the PID control or due to vibrations being transmitted to the cells through the long external tubing. It is plausible that the conditioning processes that are responsible neuronal viability require a fine temporal arrangement of factors, e.g., a stable gradient of growth factors or a precise placment of ECM proteins around an extended neurite. In this case it possible that even a seemingly minor property like flow smoothness could play a big role in the viability. We believe that it would be worthwhile to perform further, more systematic examinations of the effect of different flow regimes on neuronal viability as it will facilitate our understanding of such trophic volume transmission processes which remain poorly understood. These processes would need to be taken into account if long term neuronal survival is to be achieved and they bear relevance to the type of neuroscience experiments that may be performed with the microfluidics approach. Nevertheless, we can conclude that appropriate media conditioning allows maintenance of viable neuronal cultures for several hours.

In this work, considerable focus is directed towards justifying the notion that flow primarily interferes with factor concentrations. This is due to the fact that this narrative contradicts the current perceptions in the microfluidics field which attribute a great degree of shear stress sensitivity to neurons. Morel et. al. specified a need for shear-free agonist delivery but did not check what exactly the impact of shear is \cite{morel2012amplification}, Wang et. al. reported a shear dependent retraction of neurites \cite{wang2008microfluidics} and Liu et. al. reported that a compound isolated from brain tissue confers shear protection to cultured neurons \cite{liu2013galanin}. Although we do not deny the observations made in these studies we claim that their interpretation have been misleading. Attributing deleterious effects directly to shear implies a physical damage, e.g., tearing holes in the membrane or pulling the cell off the surface. However, as unique as they may be, neurons are still, basically, adherent eukaryotic cells composed of standard universal building blocks (e.g., lipid bi-layer). Thus it is hard to conceive that they would be more prone to physical damage than any other eukaryotic cell. Studies on detachment of adherent eukaryotic cells from the surface due to shear forces report that shear rates 2-3 orders of magnitude greater than the ones typically used in microfluidic neuronal studies are required to tear the cells off the surface (e.g., \cite{decave2002shear}). Thus it is very unlikely that physical damage is inflicted to the cells. Nevertheless, there is a viable option is that shear activates stretch receptors which in turn initiate an active cell death process. Indeed such shear sensors have been identified before, but studies involving them usually report a correlation between shear rates and the cellular phenotype \cite{toh2011fluid,shemesh2015flow} which is absent from our results. Overall, untangling pure physical shear effects from those involving localized changes to factor concentrations is very hard to do experimentally. Despite this ambiguity, a major focus is given shear stress as a driver of neuronal death or even of other induced cellular phenotypes (e.g., \cite{toh2011fluid}). This focus probably has a lot to do with the ease at which shear may be defined and measured compared to changes in localized factor concentration. The data provided above serve as a reminder that the latter type of effects are just as or even more likely to play a role.

\section{Activity under flow}
In chapter \ref{chap:activityAndFlow}, in accordance with the third goal of this work, we developed a protocol whereby the cultures exhibited stable spiking activity under flow. To develop this protocol, we monitored the activity of cultures under steady microfluidic flow in devices bonded to commercial MEAs. The final working protocol consisted of older culture (3 weeks old) under flow with media from the same culture dish (dubbed `self media'). Nevertheless, in the process of developing this protocol we realized that the effects of the flow on the activity seemed to follow exactly the same principles that were uncovered by the viability studies discussed in the previous section, namely, that the flow can cause a disruption to the activity which is primarily determined by changes to the chemistry around the cells rather than by shear stress. The observations which led to this conclusion are as follows: Under fast flow on younger cultures (2 weeks old), the activity suffered a disruption consisting an immediate loss of the stimulation response as well as a sharp de-synchronization of the spontaneous activity which was then gradually silenced altogether. Under very slow flow (\(u=3 \mu m\cdot s^{-1}\) and \(P_{e}=10\)) this disruption was completely avoided. However, the involvement of shear was refuted by repeating the fast flow experiments with a semipermeable membrane positioned \(100\mu m\) above the cells and separating them from the flow. Even though the membrane was shown to only allow negligible shear underneath (much smaller than even in the case of the slow flow above), it did nothing to prevent the activity disruption. By calculating the expected diffusive flux through the membrane assuming that the flow prevents concentrations above it from building up, we were able to show that the above immediate effects on the activity are consistent with the removal of neurotransmitter sized molecules purely by diffusive transport through the membrane (see expanded arguments in section \ref{sec:crossFlow:membrane}).

In this context it is worthwhile mentioning that Wagenaar et. al., in their influential study about activity dynamics in neuronal culture \cite{wagenaar2006extremely}, mentioned that even small physical perturbations to the culture dish (e.g., moving it across the room) can induce strong activity disturbances which last minutes. The interpretation provided by the authors was that the action of sensitive stretch receptors can generate a lasting modulation of the network dynamics. Our findings suggest that a more likely scenario is that activity disruptions associated with physical perturbations to the culture dish are, in fact, caused by a temporary mixing of the culture media which interrupts the organization of factors around the cells.


The final protocol, where the activity was maintained in 3 weeks old cultures, did not require any shear reduction measures and the key was to use self media as described above. An additional important outcome of these studies was that the two culture age groups that were studied exhibited contrasting acute effects (i.e., immediately on the onset of flow) when placed under flow with media from the other group. Older cultures under young media exhibited strong excitation whereas young cultures under old media were significantly depressed. Thus every developmental stage is characterized by an excitatory / inhibitory tone which is represented both in the culture and in the extra cellular milieu and both need to be matched. We would like to note that these tone transitions were only part of the effect because they cannot explain the observed network disfunction in its entirety, but they are taken to demonstrate the need for matching between the developmental stage of the culture and the flow media implicating self media as the best option. This age parameter was not at all part of the discussion in the case of the preceding viability studies and could potentially be also relevant in that case and so additional improvements to the viability may be achieved.

In this work we have observed two scenarios where the cultures maintain a stable network activity, either through the specialized protocol where the flow media matches microenvironment of the tested culture, or by reducing the factor removal to a level tolerable by the cells (i.e., that enables them to maintain control of their surroundings). The second option may be achieved simply by reducing the flow speed (to several microns per second) or by decoupling the flow from the cells to a great enough distance (\(1mm\)) so that factors do not get removed by the diffusion. Taken together these protocols and principles cater for a variety microfluidic flow application to neuronal culture from slow fluidic isolation and gradient generation to the rapid pulsing employed here. We expect these principles to be the basis of future device designs and to facilitate research employing this technology.



\subsection{The state of the system under rapid flow}
A noteworthy observation is that even with most successful protocol where the activity was generally maintained we still observed instability and excitability drifts developing over time (e.g., section \ref{sec:pulsing:dopamine}) implying that these experiments cannot be run for days and weeks as is customary with neuronal cultures on MEAs. On one hand this could simply mean that a conditioning protocol to generate media which fully matches the microenvironment has yet to be achieved. On the other hand it is important take into account is that the extrasynaptic microenvironment is not uniform and static as some idealized caricatures of neural tissue may suggest but rather spatially and temporally complex as it is constructed out of an assembly of ongoing localized secretion events. The calculations in section \ref{sec:crossFlow:membrane} showed that fast flow can act as a strong sink perturbing the concentration around the cells purely due to diffusion even when stream was located \(100\mu m\) away from the cells (i.e., over the membrane). With this in mind, it very plausible that direct flow, where the stream runs just on top of the culture (i.e., much closer than the membrane experiment), flattens the spatial richness of the signalling and imposes a degenerate uniform spatial profile which directly follows the concentrations present in the stream. Thus, even if the flow media is completely matched, disturbances to the cell signalling are still expected. To put it in other words, the flow media can match the microenvironment only to the level of the average factor concentrations and will therefore inevitably disrupt phasic signals (depending on how close the stream is to the cells).

The above assertion imposes constraints on the level of functionality that may be achieved under fast flow and surely should be taken into account when designing such devices. Nevertheless, we find that measurement of network activity in these conditions might actually comprise a useful paradigm for studying the importance of extrasynaptic signalling. Even though the richness of volume transmission mechanisms has been receiving increasing attention recently, the fundamental paradigm is still that the functional identity of the network is stored in its synaptic connections and that extrasynaptic processes play only a supporting role. The standard way of performing biological investigations which is highly reductionist and tends to pay attention to a single bio-molecule at a time, makes it hard to critique the above assertion. However, our system, where the flow is thought to attenuate the phasic extrasynaptic signalling, may be used to observe how the extrasynaptic compartment is involved in network function as a whole. In section \ref{sec:crossFlow:interp} we argued, by taking into account the time scales of synaptic processes, that the flow does not interfere with phasic synaptic signalling but rather only disrupts intrinsic volume transmission. This argument further strengthens the notion that rapid microfluidic flow may be used to specifically target the extrasynaptic signalling.

The results provided in this work are already interesting with respect to the link between volume transmission and network function. Firstly, we find it is surprising that the effects of flow, when the media was not matched, were so profound and seemed to drive the network outside its functional regime (inability to maintain any level of activity and abolishment of the stimulation response). This promotes the view that the fast synaptic currents in fact only modulate the neuronal activity on top of a much stronger signal that is carried by the volume transmission. Indeed this notion has been supported by the observation that inhibitory currents originating from extrasynaptic GABA receptors actually comprise the lion's share of the total inhibitory current in neurons \cite{farrant2005variations,mody2004diversity}. An accepted paradigm in contemporary neuroscience is that neuronal activity is governed by a balance between excitation and inhibition. These results raise the possibility that this balance is strongly dependent on volume transmission processes. In our experiments, once the activity was perturbed, the functionality of the circuit was not restored even after a few hours under flow (although recovery could be induced by stopping the flow). This demonstrates that the neurons lack other activity homeostasis mechanisms (e.g., intrinsic or synaptic) that can operate within hours to restore the activity. However, it was interesting that media from older cultures, despite being mismatched in terms of age group, produced improved results in younger cultures which did maintain basic network function for a while under flow with it (and not under their own self media). These results point to an extracellular factor that is required for network function that and is not exclusive to a particular stage of development. This factor might turn out to be trivial such as a neurotransmitter precursor but it could also be a novel signalling entity that has not yet been characterized. Thus, this unique system could provide an assay for identifying and testing such novel factors.

Secondly, experiments where the media was matched (older cultures and self media) provide information as to the effects of roughly maintaining the averaged extrasynaptic concentrations while attenuating their phasic components (due to the reasons explained above). Under these conditions the fundamental functional identity of the circuit was maintained in the sense that the activity measures that we used as well as the structure of the correlation matrix were roughly maintained (figures \ref{fig:crossFlow:youngOldStats}, \ref{fig:crossFlow:youngOldExampleRaster} and \ref{fig:crossFlow:moreOldExamplesRaster}). This supports the original view that intrinsic volume transmission only plays a supporting role with regards to network function. Nevertheless these experiments still exhibited a significant increase in the intensity and length of the network stimulation response and the activity was generally much more variable. Previous work has described volume feedback mechanisms whereby network excitation triggers release of inhibitory species to control the activity. Such mechanisms were observed for GABA \cite{tamas2003identified}, adenosine \cite{wall2015localized} and ATP \cite{zhang2003atp}. The noted changes in dynamics could reflect the loss of such feedback mechanisms and therefore offers a novel and interesting view of how they operate. These observation were made with only crude measures of functionality and so we propose that, given a more advanced experimental design and analysis, these type of manipulations could provide important insight as to the role of the extrasynaptic environment.


\section{The phasic neuromodulation model}
In chapter \ref{chap:microculturePulses} we delivered the declared goals of this Ph.D by putting together all the advances made in the preceding chapters and establishing a system for rapid solution exchange to an entire neuronal culture at time scales compatible with phasic neuromodulation. Achieving these time scales required using a microculture because sweeping the interface across a macroculture would have been too slow. However, before performing the work it was unclear whether the microcultures, which comprised just 400-500 cells, would indeed function as the macrocultures did. In this chapter we showed that the microcultures are well confined spatially (i.e., do not extend neurites out of the microwells), that they develop well for over 3 weeks \textit{in vitro}, that they produced spontaneous and evoked spiking activity and that they can be made to maintain stable activity under flow by using the same protocol that was effective for the macrocultures. We further constructed a finite element fluid dynamics model of the agonist pulsing and used it to show that the concentration transient at the bottom of the microwell adheres to the required time scales. We demonstrated how this model may be used to obtain a detailed spatial profile of the concentration transients and to design other pulsing systems that meet bespoke criteria. Beyond the model based validation, we directly demonstrated adherence to the required time scales by pulsing glutamate and monitoring the excitatory response.

The establishment of a functional agonist pulsing system allowed us to design and run an experimental paradigm modeled after the Izhikevic thought experiment (section \ref{sec:introduction:izi}). The experiment consisted coupling of dopamine pulses to electrical stimulations that induce a reverberatory network response. The stimulation responses were monitored before and after the dopamine coupling epoch to assess whether lasting changes had been induced (i.e., LTP/LTD). Interestingly, the dopamine coupling induced an immediate significant depression in the stimulation responses but this effect disappeared as soon as the coupling was concluded. As described in section \ref{sec:introduction:neuromodulators} such direct modulation of the network activity by dopamine is a described phenomena. However, it is completely absent from the Izhikevic model which assumes that the effect of dopamine is just to gradually modulate the synaptic strength. We find that this discrepancy is a great demonstration of how our system may be used to ground computational theories related to neuromodulation to experimental facts. In virtue of the fine temporal control offered by our pulsing system, the observed direct response could potentially be avoided by increasing the delay between the stimulations and the dopamine reward. This would be compatible with the notion of distal reward (see sections \ref{sec:introduction:izi} and \ref{sec:pulsing:dopamine}) and could promote a scenario more compatible with the Izhikevic experiment. Nevertheless, it is currently unknown whether the processes of direct modulation and plasticity are indeed independent or if they interact in some way. Our system offers a the means of exploring these questions.

The above-mentioned results of the pulsing study contradict those obtained in the plasticity experiments performed in section \ref{sec:activity:plasticityProtocol} with bath application of dopamine. Those earlier experiments showed activity modifications which persisted after washing away the agonist. Nevertheless, the later results have been produced using the superior microfluidic based pulsing system where the concentrations levels of the agonist are accurately controlled and effects associated with the solution exchange are avoided through use of continuous flow. We are therefore inclined to believe that the earlier results are an artefact associated with residual dopamine remaining after the washing or with the washing step itself.

\subsection{Future use of the model and concluding remarks}

It is the nature of scientific progress that once you embark on what seems a desolate path, you find that it leads to a rich web of trails traversed by many other explorers. Thus I could not, should not and would not predict all the multitude of ways the system presented here could be applied. From our vantage point, this system was originally conceived with the intention of studying how the timing and concentrations of neuromodulators interact with plasticity in neuronal circuits. However, this path cannot proceed without even observing plasticity in the first place, a way point that we did not reach with the time frame of this work. As reviewed in section \ref{sec:activity:plasticityProtocol} plasticity in neuronal cultures is a controversial subject and a standard protocol for achieving it has yet to arise. It is commonly argued that since neuronal cultures mature without any form of natural input they degenerate into over-connectedness which makes it harder to induce plasticity. As we mentioned in section \ref{sec:activity:evoked}, these cultures exhibit a characteristic network response profile which seems to be fixed regardless of the locus of stimulation. This fact suggests that there is a certain connectivity pattern that is invoked whenever the network is activated. A possible interpretation is that by coupling dopamine pulses to stimulation responses we were simply reinforcing an overused circuit where the plastic connections along it had already been saturated. We therefore suggest a next step where we would apply a similar experimental paradigm but reinforce a subset of stimulation responses which contained rare events, e.g., propagation of spiking activity to an electrode which infrequently participates in the response. To achieve this the system would need to be setup to administer pulses in feedback from the activity.

This system was designed to generate rapid agonist pulses across an entire neuronal circuit and, to our knowledge, this is the first demonstration of such a capability with fast kinetic control. In the previous section, we proposed that the system may operate as switch, capable of turning off the intrinsic volume transmission in the tissue. These manipulations are just the tip of the iceberg of what could potentially be achieved with microfluidics technology. One could potentially envisage a design comprising an array of delivery channels where different agonists would be applied in a temporally and spatially controlled manner. Such a system could potentially capture the spatial complexity of localized intrinsic volume transmission processes and provide neuroscience with a valuable tool of investigating information processing in the extrasynaptic compartment, a major part of the system so far neglected due to lack of experimental accessibility. Exploration of such multi channel, high resolution agonist delivery systems has already begun \cite{scott2013microfluidic}. We hope that our work will provide valuable information as to how such systems may be combined with neuronal culture. The sky is the limit.

\vspace{1cm}
``All that we know about the future is that it is going to be different.'' Peter Drucker. 